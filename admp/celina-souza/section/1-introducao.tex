\section{Introdução}

Resumo do artigo: Políticas públicas - uma revisão da literatura por~\citeonline{souza2006politicas}.

A sua importância ressurgiu nas últimas décadas. 
Assim, estabelecendo regras e modelos que regem a decisão de políticas públicas em instituições, sendo fundamentada por fatores.  
O primeiro fator são as políticas restritivas de gastos, essas sendo baseadas pelo Consenso de Washington, o qual passaram a dominar a agenda de diversos países, como o Brasil.
O segundo fator foi a substituição das políticas keynesianas para políticas liberais.

O objetivo do artigo é descrever a literatura clássica e a mais recente sobre políticas públicas. 
Além disso, busca construir algumas pontes entre as diferentes vertentes das teorias neo-institucionalistas e análise de políticas públicas e contribuir para o teste empírico das pesquisas sobre políticas públicas brasileiras. 
