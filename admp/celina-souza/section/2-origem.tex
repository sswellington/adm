\section{Surgimento da área de políticas públicas}


Em 1936 Harold Lasswell usou o termo policy analysis de forma a conciliar o conhecimento científico com a produção empírica dos governos e também de maneira a estabelecer o diálogo entre os cientistas sociais e os grupos de interesses. 
Então, em 1497 Herber A. Simon sustentou que a tomada de decisão é o centro vital da administração, por isto, deve se basear em processos decisórios considerando a lógica e a psicologia da escolha humana~\cite{rua2013}.


A área de políticas públicas se desenvolveu em campos de estudo a partir de 1950. 
A aplicação da Teoria dos sistemas por David Easton em 1953.
O livro The Policy Sciences de Lasswell e Daniel Lerner, que propôs o termo policy sciences, desdobra o processo de produção da política pública, sugerindo as etapas de informação, promoção, prescrição, invocação, aplicação, término e avaliação. 
Esses pensamento foram críticados por Charles E. Lindblom em 1959, que defendiam o processo político e decisório, constitui um processo interativo e complexo, sem início ou fim, no qual as relações de poder é o elemento crucial~\cite{rua2013}.

A década de 1960 foi marcada pelo Modelo de Análise Sistêmica por Easton em 1962, que resultou em polêmicas entre o pensamento elitista e a abordagem pluralista. Além disso, Easton definiu em 1965 a policies como produto da operação da politics~\cite{rua2013}.

As décadas seguintes tornaram a pauta para formação da agenda, recebendo contribuição de Richard Rose em 1973. 
Por fim, os estágios da policy cycle foram elaborados, as contribuições de Jeffrey Pressman, Aaron Wildavsky, Carol Weiss entre outros~\cite{rua2013}.

Conforme Lowi, as políticas públicas são regras formuladas por autoridades governamentais que expressam intenção de influenciar o comportamento dos cidadãos por intermédio da utilização de sanções positivas ou negativas. 
Portanto, os governos coagem os seus cidadãos por meio dessas sanções~\cite{rua2013}. 

Conclui-se que o pressuposto analítico visa a formulação científica e análise por pesquisadores independentes. 
O estudo das instituições considera fundamentais para limitar a tirania e as paixões inerentes à natureza humana, conforme a tradição de Madison. 
As organizações locais são as virtudes cívicas para promover o governo ideal, de acordo com a tradição de Paine e Tocqueville. 
Política pública é um ramo da ciência da política para entender como e por que os governos optam por determinadas ações.
