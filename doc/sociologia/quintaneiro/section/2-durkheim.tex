\newpage
\section{Émile Durkheim}\label{durkheim}

O francês nasceu em 15 de abril de 1858 e faleceu em 15 de novembro de 1917, contribuiu para a consolidação da Sociologia como ciência empírica, obteve influência da filosofia racionalista de Kanto, do Darwinismo, do organicismo alemão e do socialismo de cátedra.
Portanto via a ciência social uma expressão de consciência racional das sociedades modernas, porém desprezava as áreas da história, economia e psicologia, embora descrevesse elas na explicação social.


\subsection{A Especificidade do objeto sociológico}

Para Durkheim a sociologia é a ciência das instituições, da sua gênese e do seu funcionamento, ou seja, toda a crença, todo comportamento instituído pela coletividade. 
Portanto, segundo a ordem de problemas a que se dedique, a Sociologia poderia ser dividida em Morfologia Social, Fisiologia Social, Sociologia Religiosa, Moral, Jurídica, Econômica, Linguística, Estética e, por fim, a que sintetiza suas conclusões, a Sociologia Geral. 
O ramo da Sociologia que se dedica a estudar os fatos morais, por exemplo, corresponde à “razão humana aplicada à ordem moral, inicialmente para conhecê-la e compreendê-la, em seguida para orientar suas transformações”, sempre cuidando de afastar os sentimentos pessoais. 
Essa alta consciência só pode ser adquirida pela ciência que é, ela mesma, uma obra social.
Assim, precisava delimitar o objetivo dos fatos sociais.

Fatos podem ser menos consolidados, mais fluidos, são as maneiras de agir, podendo ocorrer variação de acordo com o tempo e o local de modo coercitivo e externos aos indivíduos.
Por encontrar-se fora dos indivíduos e possuir ascendência sobre eles, consistem em uma realidade objetiva.
A influência do caráter externo é demonstrado ao comparar a forma de ensino a que é submetida uma criança, que ao passar do tempo elas vão adquirindo os hábitos que lhes são ensinando e deixando de sentir-lhes a coação. 
As representações coletivas demonstram a forma que a sociedade vê a si mesma e ao mundo que a acerca.
Valores de uma sociedade possuem uma realidade objetiva, independente do sentimento ou da importância que alguém individualmente lhes dá.


\subsection{O método de estudo da sociologia segundo Durkheim}

A Sociologia examina os fatos de maneira a perceber as mudança de valores sobre o que é normalidade, anormalidade e criminalidade, ou seja, as possíveis relações de causa e efeito e regularidades com vistas à descoberta de leis e mesmo de regras que ainda serão criadas, assim observando os fenômenos rigorosamente definidos. 
Logo, a Sociologia estabelece etapas, que contempla a assistir aos fatos sociais, por exemplo o aspecto exterior da sociedade, que é constituída por uma população em um determinado território.


\subsection{A dualidade dos fatos morais}

As regras morais são coisas agradáveis e as desejamos de maneira espontânea.
Assim, a coação é substituída pelas regras morais, porém a coação ainda existe em forma das instituições.
Essas apresentam-se de modo dualista, pois elas probi, que não se ousa violar, mas é também o ser bondoso e almejado. 
Ou seja,  ao mesmo tempo que as instituições se impõem aos indivíduos, aderem a elas.

Há necessidade de revigorar os ideais coletivos como a razão de diversos ritos religiosos que voltam a reunir os fiéis, antes dispersos e isolados, para fazer renascer e alentar neles as crenças comuns.
De maneira semelhante a sociedade  refaz-se moralmente, reafirma os sentimentos e ideias que constituem sua unidade e personalidade, assim, renovando a coesão.


\subsection{Coesão, solidariedade e os dois tipos de consciência}

Durkheim estabelece que o ser humano possui dois tipos de consciência, a individual e a outra coletiva. 
Este não representa a nós mesmos, mas a sociedade agindo e vivendo em nós. 
Aquele representa as nossas particularidades, o que nos faz indivíduo.

O objetivo da instrução pública é constituir a consciência comum, formar cidadãos para a sociedade e não mão de obra. 
Portanto, o ensino precisa ser moralizador, libertar os espíritos das visões egoístas e dos interesses materiais.
Além disso, substituir a piedade religiosa por uma espécie de piedade social.
Destarte, produz um mundo de sentimentos e de ideias e independe das manifestações individuais, pois a realidade é própria e de outra natureza, à medida que os membros do grupo sintam-se atraídos pelas similitudes uns dos outros, ao mesmo tempo que a sua individualidade é menor, ou seja, quanto mais o meio social se amplia, menos o desenvolvimento das divergência privadas é contido. 
Ainda assim, a consciência moral da sociedade não é encontrada em todos indivíduos.


\subsection{Os dois tipos de solidariedade}

Os laços que unem os membros entre si e ao próprio grupo constituem a solidariedade, a qual pode ser orgânica ou mecânica. 
Isto de acordo ao tipo de sociedade e a coesão que busca garantir. 

Os vínculos assemelham-se aos que ligam um déspota aos seus súditos, de maneira à dos laços que unem um proprietário e seus bens, que uma forma mecânica.
Portanto, ao definir mecânico é quando liga diretamente o indivíduo à sociedade, sem nenhum intermediário, constituindo-se um conjunto que aproximadamente organizado de crenças e sentimentos comuns a todos os membros do grupo.

A horda é a integração social simples sem hierarquia, que depende da extensão da vida social que ela abrange e que é regulamentada pela consciência comum.
O clã é estabelecido quando a sociedade unifica as horas, de modo a torná-las complexas.
Por conseguinte,  forma uma sociedade polisegmentar simples agregado homogêneo, de natureza familiar e política, fundado em uma forte solidariedade mecânica.
As sociedades parciais são a dissolução das sociedade segmentares, no processo se aproxima os membros que formam a vida social generaliza-se em lugar de concentrar-se em uma quantidade  de pequenos grupos distintos e semelhantes, sendo que contribuir para o aumento da densidade moral e dinâmica.
As cidades são construídas com a concentração das sociedades parciais, que aumentam a densidade da sociedade à medida que multiplica as relações intersociais, resultando em divisão do trabalho social, a solidariedade mecânica se reduz, à proporção que é substituída pela solidariedade orgânica, a fim de integrar o corpo social, assegurar-lhe a unidade.

Durkheim define indivíduos no sentido moderno da expressão quando se vive em uma sociedade altamente diferenciada, ou seja, no qual a divisão do trabalho está presente, e na qual a consciência coletiva ocupa um espaço já muito reduzido em face da consciência individual.  


\subsection{Os indicadores dos tipos de solidariedade}

Durkheim se inspira nas normas do direito, em virtude que é uma maneira estável e precisa, que simboliza os elementos mais essenciais da solidariedade social.
As sanções impostas pelo costume, as que impõem por intermédio do direito são organizadas.
Essas sanções são categorizadas em duas classes: as repressivas e as restitutivas.
A primeira inflige ao culpado uma punição. 
Por último, ações que levam o culpado reparar o dano causado.
Então, quem ameaça a unidade do corpo social deve ser punido a fim de que a coesão seja protegido, ou seja, a punição existe para sustentar a vitalidade dos laços que interligam os membros da sociedade, evitando que se relaxem e debilitam, assim, a solidariedade que mantém unidos os membros. 
Importante, que as regras morais são mutáveis.

Suicídio é um ato da pessoa e que ó a ela atinge, tudo indica que deva depender exclusivamente de fatores individuais. No entanto, Durkheim descreve uma tipologia dos suicidas: egoísta, altruísta e o anômico, 
Essas são influenciadas por fatores externos ao indivíduo, como condições sociais, profissões, meio-ambiente,  ou religião.


\subsection{Moralidade e anomia}

O mundo moderno caracterizar-se-ia por um redução na eficácia de determinadas instituições integradoras como a religião e a família, já que os indivíduos passam a agrupar-se segundo suas atividades profissionais. 
Entretanto, sob certas circunstâncias a divisão do trabalho pode agir de maneira dissolvente, deixando de cumprir seu papel moral.
Assim, a ausência de normas, que em situação normal se desprendem por si mesmas como prolongações da divisão do trabalho, impossibilita que a competição presente na vida social seja moderada e que se promova a harmonia das funções.
Então, os três casos em que isto ocorre são devidos as crises industriais e comerciais que denotam que as funções sociais não estão bem adaptadas entre si; em lutas entre o trabalho e o capital que mostram a falta de unidade e a desarmonia entre os trabalhadores e os patrões; e na divisão extrema de especialidade no interior da ciência.
Portanto, Durkheim compreende a luta de classes como uma expressão de anormalidade ao nível das relações sociais, de modo a defender o mérito do esforço pessoal possui caráter moral, então, integrador.

Dons naturais são atribuídos ao nascimento, podendo ser a inteligência, o gosto, a coragem.
Além disso, será necessário a disciplina moral para forçar os menos favorecidos pela natureza a aceitarem o que devem ao acaso de seu nascimento.


\subsection{Moral e vida social}

A moralidade é moldada de acordo com o tempo, local e povo.
A ligação do membro a um grupo é também a sua adesão a um determinado ideal social, e só na vida coletiva o indivíduo aprende a idealizar. 
Conforme afirma Durkheim: o homem não é humano senão porque vive em sociedade e sair dela é deixar de sê-lo. 
Destarte, é a sociedade que ensina aos homens a virtude do sacrifício, da privação e a subordinação de seus fins individuais a outros mais elevados.
As sociedades podem consagrar seu amor-próprio não ser as maiores, ou mais abastadas, e sim a ser as mais justas, as mais bem organizadas, a possuir a melhor constituição moral.
O Estado possui o objetivo de formar os seus cidadãos, por isto, os deveres cívicos não passarão de forma mais particular dos deveres gerais da humanidade.


\subsection{Religião e moral}

As religiões são constituídas por um sistema solidário de crenças e de práticas relativas ao sagrado, isto é, separadas, interditas, ou seja, crenças comuns a todos aqueles que se unem em uma determinada comunidade moral denominada de igreja.
Os fenômenos religiosos estão relacionados aos estados de opinião e os modos de conduta, respectivamente as crenças e os ritos.


\subsection{A teoria sociológica do conhecimento}

A religião representa a própria sociedade idealizada, reflete as aspirações para o bem, o belo, o ideal e também incorpora o mal, a morte e o mesmo os aspectos mais repugnantes e vulgares da vida social.
Ao exteriorizar sentimentos comuns, as religiões são também os primeiros sistemas coletivos de representação do mundo.


\subsection{Conclusões}

Durkheim analisa a sociedade da época concluindo que o frio moral era dado pela sociedade industrial, assim, as forças psíquicas recém-nascidas permitem aos homens recuperar o vigor de sua fé no caráter sagrado de suas sociedades e transformar seu meio, atribuindo-lhe a dignidade de um mundo ideal.

A respeito das representações coletivas e dos sistemas lógicos de compreensão do mundo originários de distintos grupos sociais estabeleceram uma ponte entre a sua teoria sociológica e as preocupações que marcam a antropologia contemporânea.
Os aspectos ligados ao consenso e a à integração do sistema social, foram incorporadas à moderna teoria sociológica norte-americana por intermédio da interpretação de Talcott Parsons. 
Além disso, também influenciou os estudos recentes sobre a desintegração de padrões tradicionais de interação devidos aos processos de urbanização e pesquisas sobre a família, a profissão e a socialização.
Logo, a sociedade é a origem para a ciência, moral e religião.

