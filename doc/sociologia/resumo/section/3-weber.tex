\newpage
\section{Max Weber}\label{weber}

Nasceu em 21 de abril de 1864 na Prússia e faleceu em 14 de junho de 1920 na Alemanha, no qual ocorriam debates sobre filosofia, pensamento social e positivismo.
Influenciado por Karl Marx e Friedrich Nietzsche.
O primeiro contribuiu para a formação de Max Weber por intermédio da perspectiva histórica, social, econômica e ideológica.
Por último, auxílio à compreensão sobre os valores antagônicos, realidade social, política e econômica, além do pluralismo democrático.
Portanto, os conceitos desenvolvidos por esses autores auxiliaram Max Weber a interpretar a complexa luta de classe e o desenvolvimento histórico \cite{quintaneiro2003toque}.


\subsection{O ascetismo e o espírito do capitalismo}

A religião influenciava na formação do cidadão e logo na forma de governo do Estado.
Sendo a religião predominante o protestantismo ascético (exercício espiritual), que é derivado do calvinismo, criado por João Calvino em 1536, no qual possui a crença que o homem já nasceu predestinado à salvação ou condenação. 
Assim, a burguesia se apoiou em tal movimento proteste devido a tese da predestinação, que justificava a concentração de renda como uma bênção do céu e considerava o trabalho como sintoma da graça celestial. 
O catolicismo por intermédio de Tomás de Aquino definiu o trabalho como uma necessidade para sobrevivência e manutenção do indivíduo e da comunidade, ou seja, para satisfazer as necessidades humanas. 
Destarte, criando uma ruptura entre o pensamento protestante e o catolico sobre a definição de trabalho. 
Logo, a diferenciação do conceito de trabalho entre as vertentes cristãs, que aparentemente se demonstrava sútil, resultou em mudanças sociais e com o desenvolvimento econômico que iniciou na Escolástica \cite{weber}.

A vocação para o trabalho é determinada pela ética quaker, que guia a vida do homem pela sua vocação em exercício de virtude ascética, sendo exprimida pela consciência e o zelo.
Deus não requer o trabalho em si, mas o trabalho racional na vocação.
Entretanto, a ideia puritana de vocação e de prêmio na conduta ascética estava limitada ao modo de vida disposto pelo capitalismo, de encontro aos prazeres da vida, pois acreditava que isto afastava do trabalho de sua vocação e também da religião.  
Portanto, o ser humano tendeu a reter atenção em seu trabalho e passou a poupar, principalmente a burguesia, que acumulou enorme riquesa \cite{weber}.

A religião influenciou a moralidade do povo por meio de promessas pós-vida, assim moldando a classe burguesa e também o proletariado.
Justificando que a distribuição desigual de riqueza era uma disposição especial da Divina Providência.
Contudo, Países Baixos o relacionamento à religião não era tão consolidado e o que predominava eram as teorias correntes da produtividade por intermédio do pagamento de salários \cite{weber}. 

O capitalismo e a cultura moderna são fundamentadas em conduta racional baseada na ideia de vocação, ou seja, o espírito do ascetismo cristão, conforme \citeonline{weber}:

\begin{quotation}
    \textit{
        ``O puritano quis trabalhar no âmbito da vocação; e todos fomos forçados a segui- lo. 
        Pois quando o ascetismo foi levado para fora das celas monásticas e introduzido na vida quotidiana e começou a dominar a moralidade laica, desempenhou seu papel na construção da tremenda harmonia da moderna ordem econômica.''
    } - \citeonline{weber}
\end{quotation}

Além disso, existem um conjunto de fenômenos sociais que influenciam no ascetismo protestante e também precisa ser considerado para elaborar a representação social \cite{weber}.
