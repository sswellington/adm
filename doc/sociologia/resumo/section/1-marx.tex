\newpage
\section{Karl Marx}\label{marx}


\subsection{Introdução}

Nascido em Trier na Alemanha em 05 de maio de 1818 e faleceu em 14 de março de 1883 em Londres na Inglaterra, teve influência de Kant, Voltaire e principalmente Hegel, o qual defendeu que a história é vasto, inclusivo e contínua evolução \cite{marx90}.
Além disso, Hegel destacou ao propor a existência do vínculo do Estado e os seus cidadãos. 

O manifesto comunista possui a doutrina para dar condições para a emancipação do proletariado e o fim da propriedade privada e tornando-a comunitária.
Porquanto, o conceito de propriedade vinculado como bem a um indivíduo é uma construção social, que nem sempre foi assim, que é nítido ao visitar os períodos históricos da humanidade.
Além disso, previu medidas intermediárias para a reforma do sistema capitalista, como o imposto de renda progressivo, abolição do trabalho infantil, educação gratuita para todas as crianças \cite{marx90}.
Portanto, o capitalismo seria apenas uma fase da humanidade, isto é, teria fim determinado e sendo substituído por outro modelo, como o socialismo e comunismo.
Logo, as relações sociais são guiadas pela economia, assim a vida ideológica e intelectual de uma sociedade é determinada por essas relações, assim confirmando que a inserção do comunismo modificaria as relações sociais.

\begin{quotation}
    \textit{``O modo de produção da vida material determina o caráter geral dos processos sociais, político e intelectual da vida. 
    Não é a consciência dos homens que determina sua existência; é, ao contrário, sua existência social que determina sua consciência''} (Karl Marx)
\end{quotation}

\subsection{O comunismo  - Produção do próprio mode de trocas}

O comunismo se destaca entre outros sistemas devido ao propor alteração da estrutura social e econômica, por intermédio da criação material das condições dessa estrutura.
A forma material depende das formas: intelectual, política, religiosa e social, que estabelecem o processo histórico de uma nação e povo.
Assim, a contradição é dada ao decorrer do período histórico, que corresponde a limitação efetiva, isto é, uma falha que se apresentará no futuro, de modo a gerar um fato contraditório, que será atribuído também ao passado.
Essas contradições se tornam entraves que se multiplicam e persistem na sociedade que apenas poderão ser superados por meio da revolução, pois tais confradições possui a sua origem pelo conflito das forças produtivas e o modo de trocas \cite{marx_ideologia}.

O proletariado deve garantir a sua existência, assim, apropriar-se da produção e das propriedades que estão concentradas nos proletários. 
De modo, a revolução busca a igualdade e consequentemente as classes sociais, tal qual o proletariado se despoja dos meios de produção e a propriedade privada transforma em propriedade comum a todos \cite{marx_ideologia}.


