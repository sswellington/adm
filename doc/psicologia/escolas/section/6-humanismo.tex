\newpage
\section{Psicologia Humanista}\label{humanismo}

Surgiu em reação à escola Behaviorismo e Psicanálise. 
A psicologia humanista define a pessoa como o centro de estudo, não o seu comportamento, de modo a ressaltar a liberdade do homem, em oposição ao controle como é feito pelo Behaviorismo.
Também critica a psicanálise, pois defende que os seres humanos são seres conscientes e enfatiza a espontaneidade e o papel do criador do ser humano.

O principal fator promotor do desenvolvimento da personalidade é uma tendência inata dos seres humanos para a auto-realização. 
As pessoas que vivem todo o seu potencial são aquelas que vivem plenamente a cada momento, deixando-se guiar por seus próprios instintos, em lugar de levar em consideração opiniões alheias. 
Assim, as pessoas de pensamento livre e alta criatividade.
Também defendeu o autoconceito como um padrão organizado e consciente das características de cada um desde a infância, à medida que novas experiências surgem, os conceitos podem ser substituídos ou reforçados. 
Portanto, a capacidade do indivíduo de modificar consciente e racionalmente seus pensamentos e comportamentos, fornece a base para a formação de sua personalidade. 
