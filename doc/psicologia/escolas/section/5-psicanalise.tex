\section{Psicanálise}\label{psicanalise}

Sigmund Freud é o fundador desta escola e possui o objetivo em distinguir as três zonas do funcionamento mental: o inconsciente, o consciente, o subconsciente, que são utilizados para fundamentar o estudo do ``Id'', ``Ego'' e ``Superego''. 
O ``Id'' são os conteúdos do inconsciente, isto é, situam-se as representações inacessíveis voluntariamente, os conteúdos aparecem disfarçados nos sonhos, nos atos falhos, nos sintomas, mas nunca em forma pura. 
Já, ``Ego'' administra a relação do indivíduo com o meio, coordenando os seus processos mentais e dando-lhe a unidade de uma identidade. Os conteúdos conscientes e subconscientes são acessíveis voluntariamente pelo sujeito. É possível para ele saber razoavelmente o que está pensando ou sentindo a respeito dos conteúdos que estão em sua consciência.
Por último, o ``Superego'' é a instância psíquica responsável pela censura gratuita aos conteúdos, que é construída pela moralidade e os padrões sociais \cite{silva2007psicologia_educacao}.

A Psicanálise preocupa-se com o funcionamento do inconsciente, diferentemente de outras linhas psicológicas que se debruçam sobre o funcionamento da mente tomando-a exclusivamente como consciência.
Porquanto, devido ao funcionamento do ``Id'', os seres humanos estariam sempre em busca de satisfação para seus impulsos, pois o ``Id'' ignora juízos de valor, a moral, o bem e o mal, sendo regido pelo princípio do prazer.
Então, o ``Ego'' desempenha o papel de mediador entre o ``Id'' e o mundo exterior, sendo regido pelo
princípio da realidade, assim, toda vez que o ``Ego'' se sente ameaçado pelas exigências de satisfação
imediata do ``Id'' produz a sensação de angústia, de modo que  ``Id'' e/ou ``Ego'' conflitam com o ``Superego''.

A psicossexual possui o objetivo de compreender os aspectos subjetivos do desenvolvimento e do psiquismo humano é um direcionador para os especialista, terá a missão identificar conflitos que podem estar influenciando na vida da pessoa e a auxiliar o entendimento da mente humana, e logo a natureza do ser humano e sua essência.
Além disso, a psicossexual é baseada nos estudos da psicanálise \cite{silva2007psicologia_educacao}. 


---

A psicanálise é uma escola de psicologia fundada por Sigmund Freud. Esta escola de pensamento enfatizou a influência da mente inconsciente no comportamento.

Freud acreditava que a mente humana era composta por três elementos: o id, ego e superego . O id consiste em instintos mais primitivos, enquanto o ego é o componente da personalidade acusado de lidar com a realidade. O superego é a parte da personalidade que mantém todos os ideais e valores que internalizamos de nossos pais e cultura. Freud acreditava que a interação desses três elementos foi o que levou a todos os comportamentos humanos complexos.

A escola de pensamento de Freud foi enormemente influente, mas também gerou um debate considerável. Esta controvérsia existia não só em seu tempo, mas também nas discussões modernas das teorias de Freud. Outros grandes pensadores psicanalíticos incluem:

Anna Freud
Carl Jung
Erik Erikson.
