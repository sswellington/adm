\section{Psicanálise}\label{psicanalise}

Nasceu na Áustria por Sigmund Freud~(1856-1939), sendo o fundador desta escola formado em medicina e fundamentou os estudos do inconsciente no comportamento.
Para compreender o inconsciente e impacto no componente, Freud segmentou o estudo em três subáreas de interesse tendo objetivo de distingui-las em zonas do funcionamento mental, tais subáreas são: o inconsciente, o consciente, o subconsciente, que são utilizados para fundamentar o estudo do ``Id'', ``Ego'' e ``Superego''~\cite{hothersall1997historia}. 
O ``Id'' são os conteúdos do inconsciente, isto é, situam-se as representações inacessíveis voluntariamente, os conteúdos aparecem disfarçados nos sonhos, nos atos falhos, nos sintomas, mas nunca em forma pura, ou seja, representa os instintos mais primitivos. 
Já, ``Ego'' administra a relação do indivíduo com o meio, coordenando os seus processos mentais e dando-lhe a unidade de uma identidade. 
Os conteúdos conscientes e subconscientes são acessíveis voluntariamente pelo sujeito, sendo o componente da personalidade acusado de lidar com a realidade, assim, é possível para o indivíduo saber razoavelmente o que está pensando ou sentindo a respeito dos conteúdos que estão em sua consciência.
Por último, o ``Superego'' é a instância psíquica responsável pela censura da expressão dos pensamentos, que é construída pela moralidade, cultura e os padrões sociais de determinada região
Portanto,Freud defendia a tese que esses três princípios que regem o comportamento huamno~\cite{silva2007psicologia_educacao}.

A Psicanálise preocupa-se com o funcionamento do inconsciente, diferentemente de outras linhas psicológicas que se debruçam sobre o funcionamento da mente tomando-a exclusivamente como consciência.
Porquanto, devido ao funcionamento do ``Id'', os seres humanos estariam sempre em busca de satisfação para seus impulsos, pois o ``Id'' ignora juízos de valor, a moral, o bem e o mal, sendo regido pelo princípio do prazer.
Então, o ``Ego'' desempenha o papel de mediador entre o ``Id'' e o mundo exterior, sendo regido pelo
princípio da realidade, assim, toda vez que o ``Ego'' se sente ameaçado pelas exigências de satisfação
imediata do ``Id'' produz a sensação de angústia, de modo que  ``Id'' e/ou ``Ego'' conflitam com o ``Superego''.

A psicossexual possui o objetivo de compreender os aspectos subjetivos do desenvolvimento e do psiquismo humano é um direcionador para os especialista, terá a missão identificar conflitos que podem estar influenciando na vida da pessoa e a auxiliar o entendimento da mente humana, e logo a natureza do ser humano e sua essência.
Além disso, a psicossexual é baseada nos estudos da psicanálise \cite{silva2007psicologia_educacao}. 

Por fim, é importante salientar a importância de Freud para a psicologia, porém ele não foi o único pensador nem escola, por isso, não se deve confundir a psicanálise com a psicologia, já que a psicanálise pertence a psicologia.
Além disso, os ideais defendidos por Freud possuem controvérsia entre outros pensadores, como a Escola Humanista.
