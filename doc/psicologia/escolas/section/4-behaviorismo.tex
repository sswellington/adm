\section{Behaviorismo}\label{behaviorismo}

Empregou as bases de estudo da Escola Associacionismo, assim é fundada o campo do Behaviorismo por John Watson~(1878-1958) e obteve a colaboração dos pesquisadores Ivan Pavlov~(1849-1936) e Burrhus Skinner~(1904-1990)~\cite{hothersall1997historia,sternberg2000psicologia}.
Em 1950 se tornou a principal escola da psicologia, devido ao desenvolvimento dos trabalhos dos seus pensadores, a partir da teoria ambientalista e a teoria da aprendizagem, sendo fundamentado em experimentos empíricos aplicados ao comportamento reflexo ou respondente~\cite{bock1999psicologias}.

Watson adotou os aspectos objetivos, observáveis e mensuráveis da atividade psicológica, evitando os aspectos subjetivos, pois não são mensuráveis o valor na utilidade para os conteúdos ou mecanismos mentais internos. 
Por isso, concentrou-se em estudar o comportamento observável da mente e influenciou também o estudo do comportamento, a principal vertente desta escola, o comportamento é dividido em categorias, sendo as principais: respondente, operante, involuntário e reforço.
Também adotou a experimentação em animais ao invés de humanos~\cite{bock1999psicologias,sternberg2000psicologia}.

O comportamento respondente é a ação involuntário do indivíduo, que são reação ao estímulo externo, Skinner propôs que o comportamento respondente pode ser condicionado a uma ação que gera a reação, se completando ao estudo Pavlov sobre comportamento de aprendizado involuntário.
Sendo demonstrado pelo estudo que utilizou o sino e o cachorro, no qual toda vez que alimentara o cachorro, o sino soara, de maneira que ao escutar o som sino, o cachorro começara a salivar, mesmo não sendo alimentado.
Então, cachorro criou a associação do sino ao ato ser alimentado~\cite{sternberg2000psicologia,silva2007psicologia_educacao}. 

O comportamento operante é a memória muscular, por exemplo o ato do bebê fechar os membros quando algo toca-os e ato de sugar leite materno.
Então, ao passar da fase maternal, alguns desses comportamentos operantes permanecem e outros se desenvolvem, como a região motora do cérebro, que cria a memória muscular que se mantém mesmo que o indivíduo fique anos sem praticar ação, tendo por exemplo o efeito de andar de bicicleta, patins, natação, tocar instrumento musical e outros. 
O ambiente impacta no desenvolvimento deste comportamento e também em outros comportamento e atividades~\cite{bock1999psicologias}

Reforço é tudo aquilo que aumenta a probabilidade da resposta associada anteriormente a um estímulo acontecer novamente. É consequência dos comportamentos associados à recompensa que estimulam uma ação que se repita.
O principal experimento é a caixa de Skinner idealizada por Burrhus Frederic Skinner para avaliar os  métodos propostos como: reforços, tanto positivos quanto negativos; controle de estímulos; comportamento.
Logo, comprovando as suas teses por meio de animais em ambiente controlado~\cite{silva2007psicologia_educacao}.

Os resultados dos estudos de John Watson, Ivan Pavlov e Burrhus Skinner refletem em trabalhos ainda hoje, tendo a visão ambientalista da aprendizagem por meio do comportamento observável. 
As críticas tomam a proposta de Behaviorismo estão relacionadas ao nível da generalização dos testes em animais para humanos e modelagem do comportamento como algo autoritário, que desconsidera o que se passa na mente do indivíduo, impondo o desenvolvimento dos comportamentos socialmente valorizados que implicam no arbítrio.
Apesar disso, esta escola possui influência sobre a psicologia por intermédio do fundamento das técnicas e a padronização dos experimentos utilizados nos estudos da psicologia atualmente, como o treinamento comportamental, sistemas de fichas, terapia de aversão e outras técnicas são frequentemente utilizadas em programas de psicoterapia~\cite{sternberg2000psicologia}.
