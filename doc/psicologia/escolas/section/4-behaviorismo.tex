\newpage
\section{Behaviorismo}\label{behaviorismo}

Considerado uma teoria ambientalista. 
Os behavioristas procuram estudar os aspectos objetivos, observáveis e mensuráveis da atividade psicológica, deixando de lado aspectos subjetivos, considerados não mensuráveis. 
Por isso, seus partidários deixaram de lado o estudo da mente e voltaram-se para o estudo do comportamento.

Reflexologistas sustentavam que o caminho adequado para estudar a aprendizagem era a investigação fisiológica dos reflexos. Pesquisa com maior relevância e reflexos condicionados.

Reforço é tudo aquilo que aumenta a probabilidade de uma resposta associada anteriormente a um estímulo acontecer novamente. É consequência dos comportamentos, como as recompensas, por exemplo, que costuma fazer com que eles se repitam.
O principal experimento é a caixa de Skinner idealizada por Burrhus Frederic Skinner para avaliar os  métodos de reforços, tanto positivos quanto negativos \cite{silva2007psicologia_educacao}.

Os resultados dos estudos de Skinner refletem em trabalhos ainda hoje, tendo a visão ambientalista da aprendizagem. 
As críticas tomam a proposta de Skinner de modelagem do comportamento como algo autoritário, que desconsidera o que se passa na mente do aprendiz, impondo a ele o desenvolvimento dos comportamentos
socialmente valorizados que implica na liberdade de escolha.
