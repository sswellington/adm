\section{Behaviorismo}\label{behaviorismo}

Fundada por John Watson~(1878-1958) e obteve colaboração Ivan Pavlov~(1849-1936) Burrhus Skinner~(1904-1990) \cite{hothersall1997historia}.
Em 1950 se tornou a principal escola da psicologia, devido ao desenvolvimento dos trabalhos de seus pensadores, a partir da teoria ambientalista e a teoria da aprendizagem, sendo fundamentado em experimentos empíricos.
Além disso, parte do seu desenvolvimento foi em solo norte-americano.
Os behavioristas procuram estudar os aspectos objetivos, observáveis e mensuráveis da atividade psicológica, deixando de lado aspectos subjetivos, considerados não mensuráveis. 
Por isso, seus partidários deixaram de apenas estudar a mente e adotaram também o estudo do comportamento, sendo a principal vertente desta escola~\cite{bock1999psicologias}.

Reflexologistas sustentavam que o caminho adequado para estudar a aprendizagem era a investigação fisiológica dos reflexos. Pesquisa com maior relevância e reflexos condicionados.

Reforço é tudo aquilo que aumenta a probabilidade de uma resposta associada anteriormente a um estímulo acontecer novamente. É consequência dos comportamentos, como as recompensas, por exemplo, que costuma fazer com que eles se repitam.
O principal experimento é a caixa de Skinner idealizada por Burrhus Frederic Skinner para avaliar os  métodos de reforços, tanto positivos quanto negativos \cite{silva2007psicologia_educacao}.

Os resultados dos estudos de Skinner refletem em trabalhos ainda hoje, tendo a visão ambientalista da aprendizagem. 
As críticas tomam a proposta de Skinner de modelagem do comportamento como algo autoritário, que desconsidera o que se passa na mente do aprendiz, impondo a ele o desenvolvimento dos comportamentos
socialmente valorizados que implica na liberdade de escolha.
Logo, esta escola influência  sobre a psicologia, fundamentando as técnicas e padronizando os experimentos utilizados nos estudos da psicologia atualmente, como o treinamento comportamental, sistemas de fichas, terapia de aversão e outras técnicas são frequentemente utilizadas em programas de modificação de psicoterapia e de comportamento.
