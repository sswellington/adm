\section{Estruturalismo}\label{estruturalismo}

O Estruturalismo teve impacto na inauguração da psicologia científica e a Alemanha como o seu berço, tendo a universidade de Leipzig grande contribuição com a formação e atuação de Wundt, Weber e Fechner.
Logo, psicologia definiu campo de estudo, objetivo de estudo e os seus métodos \cite{bock1999psicologias}. 

A primeira escola da psicologia criada por William James~(1842-1910) e também fundador do primeiro laboratório em 1875 de estudos de experimentos em psicofisiologia, o que faz dele os pai da psicologia como ciência , outra importante pessoa associada a esta escola foi Edward Titchener~(1867-1927), que foi o pioneiro em aplicar e nomear os estudos do Estruturalismo~\cite{hothersall1997historia,bock1999psicologias}. 
Esta escola determina área de psicologia como ciência, pois rompeu as ideias abstratas e espiritualistas, o qual defendiam a existência da alma; e adotou os princípios e métodos científicos. 

O objetivo do Estruturalismo é descobrir tudo sobre a estrutura e o conteúdo da mente humana, por meio da elementos, introspecção e associação.
Seus partidários sustentavam que cada totalidade psicológica compõe-se de elementos (sensações,imagens e sentimentos) que se encontram justapostos, associados entre si, formando assim a consciência.
A duração dessa escola foi curta, pois durou apenas o tempo de vida dos seus criados, pois outras teorias começaram a surgir e disputar o domínio na psicologia~\cite{silva2007psicologia_educacao}.  
