\section{Funcionalismo}\label{funcionalismo}

O funcionalismo foi criado por William James preocupa-se com a utilidade prática das ideias e dos comportamentos para a sobrevivência do organismo no meio ambiente e teve seu desenvolvimento nos Estados Unidos da América. 
Ao contrário de algumas das outras escolas do pensamento bem conhecidas na psicologia, o funcionalismo não está associado com um único teórico dominante. 
Em vez disso, existem alguns pensadores funcionalistas diferentes associados, incluindo John Dewey, James Rowland Angell, e Harvey Carr
Assim, possui o objetivo na adaptação, utilidade e funcionalidade dos seres vivos.
Essa busca de utilidade confere aos funcionalistas um modo pragmatista de ver o mundo, pois, para eles, só tem valor estudar os processos mentais que têm um sentido prático por meio da experiência e do comportamento, isto é, guiado pelo sentido não racional, mas movido pelas emoções \cite{silva2007psicologia_educacao}.
