\section{Funcionalismo}\label{funcionalismo}

O Funcionalismo foi criado por William James(1842-1910) preocupa-se com a utilidade prática das ideias e dos comportamentos para a sobrevivência do organismo no meio ambiente e obteve seu desenvolvimento nos Estados Unidos da América~\cite{hothersall1997historia}.

Ao contrário de algumas das outras escolas de pensamento bem conhecidas na psicologia, o Funcionalismo não está associado com um único teórico dominante. 
Em vez disso, existem alguns pensadores funcionalistas diferentes associados, incluindo John Dewey, James Rowland Angell, e Harvey Carr.
Assim, possui o objetivo na adaptação, utilidade e funcionalidade dos seres vivos por meio da consciência.
Em busca de utilidade confere aos funcionalistas o modo pragmatista de ver o mundo, porque para eles só tem valor estudar os processos mentais que têm um sentido prático por meio da experiência e do comportamento a fim de responder o motivo e o por que agem de tal forma, isto é, guiado pelo sentido não racional, mas movido pelas emoções~\cite{bock1999psicologias,silva2007psicologia_educacao}.

\citeonline{hothersall1997historia} relata que existem historiadores que desconsideram o Funcionalismo pertencente a Escola da Psicologia, pois não existe a figura do líder ao Funcionalismo, nem conjunto formal de ideia e sim o coletivo. 


\section{Associacionismo}

Defende que a concepção é formada pela associação das ideias.
Portanto, tal argumento fez do Associacionismo pai da teoria da aprendizagem na psicologia e tendo como principal representante o Edward Thorndike, por meio da Lei do Efeito.
Em suma, a Lei do Efeito reforça os comportamentos positivos como estimulante a aprendizagem e os negativos agem de forma contrária, ou seja, conjunto de associações de acordo com o estímulo e resposta ao indivíduo ao realizar uma determinada tarefa~\cite{bock1999psicologias}.


Assim, terminamos as escolas clássicas do psicologia que surgiram no século XIX, essas compartilharam propostas e autores entre si, além de despertar pensamentos críticos sobre o pertencimento às escolas ou não \cite{hothersall1997historia}.
No entanto, o Estruturalismo, o Funcionalismo, Associacionismo são pilares essenciais para a psicologia moderna, devido a contribuição sobre o métodos científico e as primeiras abordagem sobre aprendizagem e comportamento humano~\cite{bock1999psicologias}.
