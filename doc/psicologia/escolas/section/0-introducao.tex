\section{Introdução}\label{intro}


A psicologia possui o objetivo o estudo do comportamento humano, compreender a sua
essência e seus processos mentais, como demonstrado pela etimologia da palavra de origem grega, que significa o estudo(lógica) da alma(psico). 
A psicologia é campo de estudo antigo que pertencia ao campo de estudo da filosofia e ao final do século XIX e durante as primeiras décadas do século XX, a psicologia se desenvolveu tanto que precisou se tornar uma ciência específica para cumprir o seu objetivo de estudar o comportamento humano de maneira a auxiliá-lo. 
Portanto, a psicologia é a ciência com diversas aplicações na vida humana e podendo ser segmentada por setores de aplicações, tendo por exemplo: educação, relacionamento pessoal e familiar, empresarial, gestão de pessoas~\cite{freitas2008historia}.

A atuação na educação pode ser representada a forma em que o aluno se comporta e os fatores emocionais influência no meio em que ele vive, tanto no meio pessoal como escolar, assim, afetando a qualidade de vida e o seu desempenho no ambiente de ensino, conforme menciona~\citeonline{lomonaco1999psicologia_educacao} o impacto da psicologia e a sua colaboração para o desenvolvimento da educação, o qual realiza um panorama da educação junto a psicologia e como essas podem contribuir para a formação do cidadão.
Logo, a psicologia possui a meta em auxiliar compreender o ser em seu desenvolvimento pessoal e mental. 

Para compreender a origem da psicologia é necessário o estudo da filosofia e a biologia, como Hipócrates, as bases corporais; Sócrates, a forma de pensamento; Platão, mito da caverna, que demonstra um mundo de aparência; Aristóteles sobre os aspectos práticos, científicos e principalmente o pensamento racional lógica~\cite{marcondes1997filosofia}; percorrer o entendimento de ser humano do ponto de vista religioso da Idade Média, no qual diferenciava o corpo da alma, sendo o corpo o templo da alma e logo objeto sagrada e assim intocável; a construção do conceito sobre o conhecimento foi alterado ao final da Idade Média e o início do Iluminismo e surgiu o conhecimento científico; e finalmente a psicologia moderna e o surgimento das escolas da psicologia~\cite{bock1999psicologias,freitas2008historia}. 

O conhecimento do senso comum pertence a uma cultura, podendo ter alteração dos saberes de acordo com tempo e localidade.
Portanto, o senso comum evolui ao decorrer do tempo e passa ser conhecimento verdadeiro e comprovado por meio dos detentores daquela cultura.
O conhecimento científico é construído por intermédio de uma hipótese que precisa ser comprovada. 
Por conseguinte, ambos os conhecimentos são certos, assim o que distingue ambos os conhecimento é a forma, ou seja o método e os instrumentos do conhecer.
Além disso, o senso comum pode ser tornar conhecimento científico~\cite{marconi2003mep}.  

As escolas da psicologia apoiam em um eixo tendo o intuito intervir na consciência e elaborar métodos de intervenção, ou seja, pluralismo nas abordagens do comportamento humano de acordo com o ponto de vista da escola~\cite{freitas2008historia}.
As principais escolas são: a Estruturalismo, e Funcionalismo, a Psicologia da Gestalt, o Behaviorismo, a Psicanálise, o Humanismo e o Cognitivismo.
Portanto, a diferença da psicologia para a filosofia e a biologia, visto que ela possui a meta de  refletir e intervir no ser humano a fim de auxiliar em seu desenvolvimento.
