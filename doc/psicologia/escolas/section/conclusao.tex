\section{Conclusão}\label{conclusao}


A psicologia é uma ciência que busca compreender o comportamento humano, por meio dos métodos e testes.
Assim como ocorre na sociologia, apesar com o foco de estudo alterar, sendo a sociologia o campo macro, de maneira de estudar as mudanças e as relações sociais do povo \cite{quintaneiro2003toque}.
Além disso, temos a antropologia que estuda a interseção da psicologia e sociologia~\cite{castro2016antropologia}.
Portanto, demonstrando que tais ciências são correlatas e complementares, logo se a sociologia é alterado de acordo com os costumes, local e tempo, podemos assim generalizar que o estudo da psicologia também será modificada de acordo com o período observado de determinado povo, além dos métodos utilizados serem aprimoradas ao longo do tempo, de modo que possam ser utilizados na atualidade, exemplo são as Escolas em Psicologia, que no passado os pensadores se identificaram exclusivamente que uma única escola sendo a correta e atualmente possui a metodologia ecléticas para o pensamento sobre tais Escolas da Psicologia~\cite{spink2011psicologia_social}.
