\section{Conclusão}\label{conclusao}

A psicologia é a ciência que busca compreender o comportamento humano, por meio dos métodos e testes.
Assim como ocorre na sociologia, apesar com o foco de estudo alterar, sendo a sociologia o campo macro, de maneira de estudar as mudanças e as relações sociais do povo \cite{quintaneiro2003toque}.
Além disso, temos a antropologia que estuda a interseção da psicologia e sociologia~\cite{castro2016antropologia}.
Portanto, demonstrando que tais ciências são correlatas e complementares, logo se a sociologia é alterado de acordo com os costumes, local e tempo, podemos assim generalizar que o estudo da psicologia também será modificada de acordo com o período observado de determinado povo, além dos métodos utilizados serem aprimoradas ao longo do tempo, de modo que possam ser utilizados na atualidade, exemplo são as Escolas em Psicologia, que no passado os pensadores se identificaram exclusivamente que uma única escola sendo a correta e atualmente possui a metodologia ecléticas para o pensamento sobre tais Escolas da Psicologia~\cite{spink2011psicologia_social}.

As escolas da psicologia são essenciais para compreender o surgimento é a pluralidade de ideias presentes nelas e a forma que impacta na percepção do ser humano em diferentes perspectivas \cite{freitas2008historia}.
Destarte, as escolas do século XIX, Estruturalismo, Funcionalismo e Associacionismo possuíam o objetivo nítido de compreender ser humano e o seu o comportamento, porém não possuíam métodos consolidados para alcançá-lo, na medida em que os pensadores precisaram criar e desenvolver o esboço  que demonstrou um feixe do comportamento humano.
Por conseguinte, as escolas do século XX se respaldou sobre o conhecimento das escolas do século XIX para cumprir o objetivo estabelecido por estes, marcando o início da ``revolução cognitiva'' e adicionando indicadores que qualifica os processos relacionados ao processamento de informação, língua, comportamento, memória e percepção~\cite{eysenck2017psi_cognitiva}.
Assim, surgindo as escolas da Psicologia da Gestalt, da Behaviorismo e da Psicanálise.
Em relação às essas duas últimas escolas emergiram as escolas Humanista e a Cognitiva.
A humanista defende a pessoa como centro de estudo; e a Cognitiva é fundamentada pela Psicologia de Gestalt, tendo a finalidade de entender o comportamento humano e o seu estágio de desenvolvimento.
Por conseguinte, todas as escolas desfrutam do mesmo objetivo, à medida que a metodologia para alcançar o objetivo que caracteriza as escolas e a torna essas únicas~\cite{bock1999psicologias,sternberg2000psicologia}.     

Concluímos, que as escolas possuem objetivo em comum e metodologias únicas a combinação dessas permite realizar o panorama do objetivo, que ainda se encontra em aberto, apesar de que a diversos avanços para a compreensão da mente e comportamento humano e tendo esforços de diversos pesquisadores e a união de áreas tendo a finalidade de alcançar o objetivo da psicologia~\cite{sternberg2000psicologia,eysenck2017psi_cognitiva}.
Além disso, a psicologia permite a atuação interdisciplinar e contribui em diversos setores como mencionado no Capítulo~\ref{intro}. 
Na administração pública a psicologia representa como sendo uma metodologia eficaz para a compreensão do ser humano, contribuindo para a formação de equipes, a gestão de pessoas e principalmente para auxiliar o desenvolvimento humano da entidade.
