\section{Psicologia Cognitiva}\label{cognitivismo}

A meta do estudo os processos mentais e antagonista a escola behaviorismo, sendo influenciado pelo pensador Jean Piaget~(1896-1980) por meio do estudo dos estágios de desenvolvimento cognitivo. 
A escola da psicologia cognitiva pode receber nomenclatura diferente de acordo com o autor, tendo por exemplo  \citeonline{bock1999psicologias} utiliza o denominação de psicologia do desenvolvimento. 
Neste trabalho será adotado o termo para nomear a escola de psicologia cognitiva, consoante \citeonline{sternberg2000psicologia,eysenck2017psi_cognitiva}.

Os métodos empregados pela psicologia cognitiva estão em pleno desenvolvimento e aplicações, visto que estuda a forma que cognitiva é desenvolvida no ser humano, tendo diversos testes aplicações e estão constante evolução e utiliza meio tecnológicos para observar o desenvolvimento cerebral relacionado a determinadas atividades realizadas pelo homem, a fim de mapear as regiões e como elas são estimuladas, a fim de entender a motivação que levou o indivíduo tomar uma ação~\cite{nicolelis2015cerebro}. 
Portanto, as áreas de atuação são interdisciplinares, envolvendo a psicologia, neurociência, medicina, computação, linguística, filosofia e outras \cite{eysenck2017psi_cognitiva}.

A psicologia cognitiva começou em 1950, tendo objetivo em responder às afirmações propostas pela escola behaviorista, que não conseguia demonstrar o comportamento do ponto de vista neurológico.
Então a escola da psicologia cognitiva nasceu com o objetivo de responder tal lacuna deixada pelo behaviorismo~\cite{sternberg2000psicologia, laruse2009geral}.

Jean Piaget desenvolveu o estudo que mapeou os estágios de desenvolvimento cognitivo tendo o objetivo em responder o como e por que o indivíduo age de determinada maneira.
Então, como os estágios impactam no desenvolvimento pessoal, desde a influência externa ao indivíduo, como hereitariedade e meio em que ele vive afetam os aspectos do seu desenvolvimento~\cite{bock1999psicologias}. 

O estudo sobre o funcionamento da memória utiliza o estágio cognitivo de Piaget, para fundamentar a forma como a memória reconhece e armazena padrões.
Formulando o modelo de três fases proposto por Richard Atkinson e Richard Shiffrin em 1968, sendo Memória sensorial, memória de curto prazo e memória de longo prazo~\cite{laruse2009geral}.
Tal estudo será base para a compreensão das áreas do cérebro relacionada à aprendizagem, como o córtex entorrinal, hipocampo e o neocórtex, sendo responsável respectivamente por ser um sensor para selecionar os estímulos (informação) recebidos; formar a memória de curto prazo; e a memória de longo prazo~\cite{carey2014aprendemos}.
Destarte, compreendendo a região do cérebro que é formado a memória e os estágio de desenvolvimento cognitivo, permanece a lacuna sobre a motivação pela ação~\cite{sternberg2000psicologia}.

Atualmente, a Psicologia Cognitiva possui as abordagens da cognição humana de modo interdisciplinar, tendo o objetivo de compreender e aperfeiçoar os estudos sobre o comportamento humano.
Os instrumentos utilizado nesta escola envolvem a percepção dos sentidos sensoriais e a atenção, como a reconhecimento de padrões, percepção sobre movimento e ação; memória, como é formado e os tipos memória relacionadas a aprendizagem e o esquecimento;
linguagem, tendo por exemplo a percepção, compreender e a produção; e o pensamento e raciocínio~\cite{eysenck2017psi_cognitiva,sternberg2000psicologia}.
 
A Psicologia Cognitiva possui diversos desafios que precisam ser superados para o avanço sobre a compreensão do homem.
Além disso, perceba que o escopo de estudo ficou tão complexo que é necessário realizar a intercessão de diversas áreas do conhecimento, podendo no futuro ser segmentada em uma nova área do conhecimento para avançar nos estudos a fim de alcançar o objetivo, que ainda se encontra em aberto.
No futuro a Psicologia poderá passar pelo mesmo fenômeno da que ocorreu com a Filosofia, tendo o primeiro estado a concentração de diversas áreas entorno da Psicologia, de modo a ser visto como apenas uma área denominada da Psicologia, porém área se tornará tão densa que será desmembrada em outras ciências, então passando pelo mesmo processo ocorrido a Filosofia~\cite{marcondes1997filosofia}.

