\section{Psicologia Cognitiva}\label{cognitivismo}


---

A psicologia cognitiva é a escola de psicologia que estuda os processos mentais, incluindo como as pessoas pensam, percebem, lembram e aprendem. Como parte do maior campo da ciência cognitiva, este ramo da psicologia está relacionada a outras disciplinas, incluindo neurociência, filosofia e linguística.

Psicologia cognitiva começou a surgir na década de 1950, em parte como resposta ao behaviorismo. Os críticos do behaviorismo observaram que ele não conseguiu explicar o comportamento como os processos internos afetados. Este período é muitas vezes referido como a “revolução cognitiva”, com uma riqueza de pesquisa sobre temas como processamento de informação, língua, memória e percepção começando a surgir.

Uma das teorias mais influentes desta escola de pensamento foram os estágios de desenvolvimento cognitivo da teoria proposta por Jean Piaget.