\newpage
\section{Psicologia da Gestalt}\label{gestalt}


Contribuiu para o estudo sistemático do processo psicológico da percepção e procurou esclarecer o papel deste processo na organização da aprendizagem.
De modo que as teorias associacionistas vêem os fenômenos psicológicos como resultantes da soma de pequenas sensações, reações, percepções, enfim, de um somatório de partes que se combinam de maneira mecânica \cite{silva2007psicologia_educacao}.

A utilização do método científico por meio da observação de um problema que gera uma hipótese, que será provada pela experimentação que sustenta a teoria. 
Destarte, Wolfgang Kohler para sustentar os seus estudos sobre o desenvolvimento da inteligência e da
aprendizagem, utilizou chimpanzés em situações de resolução de problemas práticos e, dos resultados obtidos com estes animais, derivou idéias sobre a aprendizagem humana. 
Por conseguinte, o resultado do estudo nomeou de insight, que é o momento em que o campo perceptivo se reorganiza frente a um conhecimento novo. 
Importante ressaltar que em uma primeira observação não é possível determinar ao certo, porém ao ser analisado a solução é trivial.
Portanto, a Psicologia da Gestalt defende o isomorfismo entre os fenômenos psíquicos e os
processos cerebrais a eles subjacentes, ou seja, o funcionamento psicológico apresenta propriedades semelhantes ao funcionamento cerebral \cite{silva2007psicologia_educacao}. 
Logo, o processo de aprendizagem tende a variar de pessoa para a pessoa, a utilização dos
conceitos na Gestalt, no entanto obedecem a regras de percepção sensoriais comuns a maioria das pessoas conforme a sua faixa etária, essas são percepções baseadas em sua maioria nos aspectos fisiológicos, isto é, independente de influências externas como papel social, econômico ou cultural.
