\section{Psicologia da Gestalt}\label{gestalt}

A escola da psicologia baseada na ideia de que experimentamos as coisas de modo totalmente unificado, e não a quebra dos pensamentos e comportamentos em elementos menores, os psicólogos da Gestalt acreditam que é necessário realizar o panorama da experiência por meio das sensações do ambiente. 
De acordo com os pensadores da forma, o todo é maior do que a soma das suas partes \cite{bock1999psicologias}.

Fundada por Ernst Mach~(1838-1916) e Christian von Ehrenfels~(1859-1932) sobre a proposta de compreender o ser humano em um todo, ou seja, o todo é maior que a soma das suas partes.
O surgimento desta escola foi na Europa, sendo desenvolvida em dois países, a Alemanha e a Áustria, ainda durante o século XIX em resposta às escolas do Estruturalismo e do Associacionismo \cite{bock1999psicologias}.

A contribuição para o estudo sistemático do processo psicológico da percepção e procurou esclarecer o papel dela na organização da aprendizagem.
De modo que as teorias associacionistas vêem os fenômenos psicológicos como resultantes da soma de pequenas sensações, reações, percepções, enfim, de um somatório de partes que se combinam de maneira mecânica proposta por Max Wertheimer~(1880-1943), Wolfgang Köhler~(1887-1967) e Kurt Koffka~(1886-1941)\cite{hothersall1997historia}.

A princípio analisaram a percepção e a sensação relacionada ao movimento, de maneira a compreender os processos psicológicos correlacionados à ilusão de ótica, à proporção que os estímulos físicos são percebidos pelo indivíduo, em comparação ao estímulo real \cite{bock1999psicologias}.
Já para a Escola Gestalt é necessário compreender o homem por completo e não apenas as partes, assim se provou a escola coerente e coesa em seu estudo para compreender o ser humano~\cite{silva2007psicologia_educacao}.

A utilização do método científico por meio da observação do problema resulta em uma hipótese, que será posto à prova pela experimentação que sustenta a teoria. 
Destarte, Wolfgang Kohler para sustentar os seus estudos sobre o desenvolvimento da inteligência e da
aprendizagem, utilizou chimpanzés em situações de resolução de problemas práticos e, dos resultados obtidos com estes animais, derivou idéias sobre a aprendizagem humana. 
Por conseguinte, o resultado do estudo nomeou de insight, que é o momento em que o campo perceptivo se reorganiza frente a um conhecimento novo. 
Importante ressaltar que em uma primeira observação não é possível determinar ao certo, porém ao ser analisado a solução é trivial.
Portanto, a Psicologia da Gestalt defende o isomorfismo entre os fenômenos psíquicos e os
processos cerebrais a eles subjacentes, ou seja, o funcionamento psicológico apresenta propriedades semelhantes ao funcionamento cerebral \cite{silva2007psicologia_educacao}. 
Logo, o processo de aprendizagem tende a variar de pessoa para a pessoa, a utilização dos
conceitos na Gestalt, no entanto obedecem a regras de percepção sensoriais comuns a maioria das pessoas conforme a sua faixa etária, essas são percepções baseadas em sua maioria nos aspectos fisiológicos, isto é, independente de influências externas como papel social, econômico ou cultural.

A percepção é o ponto central da escola Gestalt, no qual o indivíduo recebe um estímulo e conseguintemente realizará uma resposta, tal fenômeno denominado de processo de percepção, que é questionada pelo princípio da escola behaviorista, que é baseada no estímulo e resposta, assim desprezando os conteúdos de ``consciência'', pois é difícil de quantificar tal variável \cite{bock1999psicologias}.
Atualmente, existem pesquisas para mapear tais conteúdos como o do pesquisador brasileiro Miguel Nicolelis por meio de seus experimentos, como o mais famoso realizado na copa do mundo de 2014 de futebol que ocorreu no Brasil, utilizou o exoesqueleto que permitiu o paraplégico caminhar, além desse trabalha a outros que realizam interface entre o homem e máquina, que permite e continuará permitindo grandes avanços na ciência e podendo quantificar tais variáveis \cite{nicolelis2015cerebro,nicolelis2020true}.
Portanto, apenas o futuro revelará os avanços dessas áreas permitirão a compreensão do conteúdo de ``consciência'', bem como ocorreram diversos avanços desde o surgimento da psicologia como ciência e principalmente decorrente dos eventos pós-guerras do século XX.
