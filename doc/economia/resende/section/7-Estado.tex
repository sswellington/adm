\section{O Estado agente de interesses próprios}

A proposta de \citeonline{piketty2013capital} para reverter a nova tendência de concentração mundial da renda é a criação de um imposto mundial sobre a riqueza.
Entretanto, \citeonline{resende2014escassez} relata que isto pode levar a fuga de capital em busca de condições mais favoráveis e o Estado já atingiu o tamanho máximo com a atual capacidade de sua gestão, de modo que o imposto não é a solução.
Sendo, uma alternativa o estudo da viabilidade de reduzir a concentração e de democratizar a propriedade do capital

\citeonline{wilkinson2010spirit} concluem que existe correlação negativa entre a desigualdade de renda e da riqueza, ou seja, a desigualdade excessiva é corrosiva, reduz a coesão social e inviabiliza a democracia.
