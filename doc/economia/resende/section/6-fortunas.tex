\section{Velhas e novas fortunas}

Nem toda a recente concentração de riqueza pode ser explicada pelo argumento de que a taxa de retorno do capital é superior à taxa de crescimento da economia.
A tese defendida por \citeonline{piketty2013capital}: se o retorno do capital é superior à taxa de crescimento econômico, a renda se concentra.
Logo, se o retorno do capital é superior à taxa de crescimento desaparece a possibilidade de novas fortunas, retornando às sociedades estáticas e estratificadas, baseadas na herança e incompatível com a democracia, em virtude que a democracia tolera a desigualdade como fruto do mérito e do trabalho, mas não como fruto do privilégio e do nepotismo.
