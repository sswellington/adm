\section{Menos crescimento, mais desigualdade}

O objetivo primordial da atividade econômica é reduzir a escassez, por intermédio do aumento da produção e da renda.
O ideal democrático das sociedades modernas estará ameaçado, se não adotar políticas especificamente voltadas para reverter o processo de reconcentração da riqueza, conforme menciona \citeonline{piketty2013capital}.
Portanto, se a taxa líquida de retorno do capital for superior à taxa de crescimento da economia, haverá um aumento da relação entre a riqueza.
Assim, o aumento da relação capital/renda implica o aumento da desigualdade. 
Tendência que só é interrompida em períodos excepcionais, turbulentos, como durante as grandes guerras ou as grandes recessões.
A teoria do crescimento adota uma definição mais restrita de capital - apenas bens que fazem parte do processo produtivo, excluindo obras de arte, jóias, propriedades residenciais, entre outras, que são certamente riqueza.
