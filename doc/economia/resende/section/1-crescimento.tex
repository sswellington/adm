\section{Crescimento: um fenômeno recente}

Desde o auge do Império Romano ao inicio do século XVIII, praticamente não houve crescimento, provavelmente estagnado.
Superando a estagnação somente no século XIX por intermédio da Revolução Industrial, obtendo $1,5\%$ ao ano.
O século XX superou o antecessor, crescendo $3\%$ ao ano. Além disso, ocorreu o crescimento populacional de $1,4\%$ ao ano.
Portanto, a população mundial passou de menos de 500 milhões, para mais de 7 bilhões de pessoas, em um espaço de três séculos.
Já no século XXI, ocorreu decréscimo econômico~($ > 2\% $) considerado inaceitavelmente baixo.
Destarte, a taxa de crescimento exponencial é insustentável ao decorrer do tempo, sem criar desequilíbrio econômico, por exemplo, o crescimento de $1\%$ ao ano, se mantido por 30 anos, o espaço de uma geração, mais do que dobra a renda dos filhos em relação a de seus pais.

Conclui-se que a taxa de crescimento demográfico mundial é parte do crescimento da renda. 
Assim, uma vez interrompido o crescimento demográfico e atingida a fronteira tecnológica, será preciso contar com o avanço da tecnologia para garantir o crescimento da produção e da renda.
