\section{Missão cumprida para Produto Interno Bruto}

Conceito abstrato adotado em 1924 para quantificar o valor aproximado da economia, antes utilizava a quantidade habitantes para expressar a economia do país. 
Simon Kuznets incrementou parâmetros tendo objetivo de quantificar toda a produção em determinado período de tempo a fim de tornar a medida mais confiável e fiel a realidade econômica.
Tendo a prioridade o setores agrícola e industrial, porém desconsiderou as transações comerciais e o setor de serviços não remunerados, ou seja, poderia ter uma falsa medida de riqueza.

\begin{citacao}
    \textit{``Se todas as donas de casa fossem contratadas para tomar conta das de suas vizinhas e usassem o que recebessem para pagar a vizinha contratada para tomar conta da sua própria casa, renda nacional teria um aumento expressivo.'' } 
    \begin{flushright}
        \citeonline{resende2014escassez}
    \end{flushright}
\end{citacao}   

Logo, ao decorrer das décadas a mudança nos setores econômicos tornou o PIB obsoleto a realidade contemporânea. 
Já que a correlação entre renda e bem estar é alta enquanto as necessidades básicas não estão atendidas, mas perde força à medida que a renda cresce e a escassez absoluta se reduz.
