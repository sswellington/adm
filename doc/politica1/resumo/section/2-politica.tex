\section{Novo modelo federativo} \label{politica}


\subsection{Introdução}
Os principais efeitos do novo modelo federativo adotado no Brasil após a Constituição de 1988, tendo  destaque aos reflexos do sistema partidário e eleitoral.


\subsection{Federalismo}

Surgiu no Brasil junto à República com a promulgação da Constituição de 1891 inspirado pelo modelo norte-americano, sendo diferenciada pela origem dos estados é formado por movimento centrífugo e a concentração de poder é centrípeta, visto que o poder está concentrado no centro, ou seja, a União \cite{matias2014curso,mendes2011branco}.
A federação garante autonomia política, que se manifesta por intermédio de 4 aptidões: auto-organização, autolegislação, autoadministração e autogoverno \cite{matias2014curso}.


A Constituição Federal de 1988 trouxe direitos e deveres para o Estado, no qual norteia os princípios fundamentais e possui a finalidade de garantir a liberdade dos cidadãos por intermédio dos direitos dos indivíduos ao coletivo e também expressa os deveres deles.
Também, expressa os direitos e deveres da política e dos partidos políticos e diversos desses direitos e deveres são expressos em leis complementares previstas pela constituinte.
Além disso, preve atos transitórios como a definição da Forma de Governo, previsto no artigo 2º do Ato das Disposições constituicionais transitórias, escolhido pelo Brasil por intermédio de Plebiscito\footnote{https://www.tse.jus.br/eleicoes/plebiscitos-e-referendos/plebiscito-1993/plebiscito-de-1993} em 1993 foi a república e a votação do Sistema de Governo que foi eleito o presidencialismo.

A Forma de Estado é organização política do território e o Brasil adota a descentralização política, assim, as entidades políticas distribuídas nos níveis nacional, regional e local, isto é, União, Estadual e Municipal.  
Tal Forma de Estado é nomeado de Federação \cite{pedro_lenza2021direito}. 

Importante para o Federalismo são instituições que garanta a estabilidade das entidades políticas, principalmente que União desrespeite a competência das outras entidades, diminuindo assim a autoridades delas e as transformando em meras unidades administrativas, por outro lado, a participação da União for reduzida à medida em que a supremacia nacional esteja ameaçada pela fragilidade da União.
Portanto, o equilíbrio das instituições e a organização política são essenciais para o progresso do Federalismo \cite{leonardo_avritzer_2006}.


\subsection{Sistema eleitoral}

O objetivo do sistema eleitoral é representar os diversos grupos da sociedade, assim, fortalecer os vínculos entre os representantes eleitos e os cidadãos.
Então, o sistema eleitoral oferece dois princípios para a escolha dos candidatos, sendo a eleição majoritária(relativa ou absoluta, tende ao bipartidarismo) e/ou proporcional \cite{leonardo_avritzer_2006}.
A eleição proporcional possui diversas regras e categorias e visam a representar os grupos sociais no parlamento de forma proporcional de seu respectivo apoio eleitoral, isto é, o número de cadeira e divido pela quantidade de eleitores e assim é estabelecido o número mínimo de votos para eleger o candidato, os votos que excederem o quociente podem ser distribuídos a outro candidato ou descartado dependendo das regras estabelecidas pelo modelo proporcional \cite{leonardo_avritzer_2006}.
Conforme \cite{leonardo_avritzer_2006} o Brasil adota o quociente Hare na primeira operação, porém usam os divisores d'Hondt para a distribuidos dos votos que excederem o quociente.


\subsection{Tipos de listas}

A lista possui duas propriedades, podendo assim ser fechada ou aberta.
A Lista Fechada é estabelecida uma lista de candidatos selecionados pelo partidos, isto é, o eleitor vota no partido.
Já, a Lista Aberta oferece ao eleitor a lista de todos candidatos, independente do partido, assim o eleitor vota diretamente no seu representante.
Conforme \cite{leonardo_avritzer_2006} o Brasil adota a lista aberta que permite os dois tipos de voto (nominal e de legenda).


\subsection{Distribuição de cadeiras}

O número de cadeiras é definido pela Lei do Cubo, que é constituida pela relação estatística extremamente forte entre o logaritmo do número de cadeiras e o cubo do logaritmo da população, tendo objetivo de representar os eleitores. \cite{leonardo_avritzer_2006}.
Conforme \cite{leonardo_avritzer_2006} há país que possui a meta de norma proporcionalidade, "uma pessoa, um voto", isto é, o voto dos cidadãos nos diferentes distritos deve ser igualmente representado na legislatura.
O Brasil não adota tal norma e adota uma norma proporcional que estabelece um número mínimo representantes eleitors a fim de incentivar o desenvolvimento nacional de regiões tendo menos eleitores. 
Além disso, existe limite máximo para os representantes por unidade federativa.


\subsection{Financiamento de campanha}

O tema está em constante evolução e mudança a fim de proporcionar um sistema que beneficie a democracia ao combate à corrupção, em visto que alguns problemas relatados por \citeonline{leonardo_avritzer_2006} foram aperfeiçoados ou alterados, porém ainda há política pública para ser implementada e aprimorada.
Atualmente, o financiamento foi alterado pela edição da Lei 13.165/2015, o qual reduziu o período eleitoral.  
O Supremo Tribunal Federal por meio Ação Direta de Inconstitucionalidade 4650, no qual determinou o fim das doações de recursos financeiros por intermédio de pessoa jurídica para campanha eleitoral.
O Tribunal Superior Eleitoral por meio da Resolução 23.463/2015 definiu parâmetros \footnote{https://divulgacandcontas.tse.jus.br/divulga/} para o financiamento de pessoa física e forma como os partidos podem receber as doações e o fundo partidário \cite{agra2019financiamento_eleitoral}.


\subsection{Orçamento participativo}

Recebeu destaque após ser adotado pelo município de Porto Alegre - RS na gestão do prefeito Alceu Collares conforme menciona \citeonline{leonardo_avritzer_2008_orc_part}, no qual a população participava das decisões orçamentárias da cidade.
Atualmente, o orçamento participativo está presente em diversos municípios e estado por intermédio de movimentos populares, associações de moradores, organizações sociais, sindicatos e associações profissionais, tendo atuação em vários setores da sociedade \cite{leonardo_avritzer_2008_orc_part}.