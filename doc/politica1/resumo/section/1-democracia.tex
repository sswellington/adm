\section{Conceito de democracia} \label{democracia}


Esta seção aborda o significado da democracia segundo~\cite{robertdahl2001}.
Também, as características que definem a democracia e como ela distinguem de outros tipos de regime. 
Por fim, a forma de governo preferível aos demais e se a democracia é plenamente realizável ou não.


A Teória Clássica de Democracia se originou aproximadamente há 2500 anos, tendo o seu local de formação ainda não definido, embora a evidências que surgiu na Grécia ou em Roma \cite{matias2014curso}.
A democracia é o governo da maioria, em alguns casos considerado como ditadura da maioria, em visto que o poder da maioria não faz o direito dessa. 
Então, perceba que a democracia não é um sistema perfeito, porém ela é a pior forma de governo, à exceção de todas as demais, parafraseando Winston Churchill.
Destarte, as razões que afirmam que esse é o melhor sistema político e os seus valores, como estabilidade e desenvolvimento das instituições \cite{robertdahl2001}.
Além disso, não existe uma democracia unânime, assim, cada modelo país e sociedade adotam o seu próprio modelo a fim de melhor representar o cenário do sistema político, instituições e as necessidades sociais de seus cidadãos, tal representação é expressa pela norma\cite{bobbio1998dicionario}.

A representação democrática é constituída de diversas formas e particularidades, desde a participação dos cidadãos até a política pública adotada pelo Estado é moldada por tal representação.
Além disso, a representação é moldada com os costumes sociais, que são modificados ao decorrer do tempo e conforme \citeonline{robertdahl2001} jamais possuiu um governo que comprisse todos os critérios de um processo democratico, assim ela não é plenamente realizável.
Além disso, ela possui diversas variações como é demonstrado por \citeonline{bobbio1998dicionario}. 
Embora, as democracias do século XXI proporcionam a participação efetiva, igualdade de voto, aquisição de informação e esclarecimento, controle sobre o planejamento sobre a política e decisões internacionais e nacionais e a inclusão \cite{robertdahl2001}.

A democracia evita sistemas tirânicos e autoritários, pois incentiva os direitos essenciais e as liberdades; determina o Estado a cumprir seus deveres com população, promovendo o desenvolvimento e igualdade; e a escolha democrática e cíclica de seus representantes \cite{robertdahl2001}.
Por conseguinte, existem instituições tendo objetivo de garantir que sistema democratico continue a funcionar e garantir o que deveres e direitos sejam respeitados, como julgar, legislar e execucar serão realizados pela repartição de poderes denominada de Tripartite criada no século XVIII por Montesquieu, pai da Teória da Democracia Moderna \cite{bobbio1998dicionario, matias2014curso}.
As democracias do século XXI, geralmente, oferecem hospitalidade aos outros países tendo o objetivo de fortalecer as relações internacionais e a diplomacia entre os países, de maneira que estimular a economia, promover as relações de paz e o desenvolvimento humano e das nações \cite{robertdahl2001}.
