\section{Teorias do ciclo de políticas pública} \label{politica_publica}


\subsection{Introdução}
Esta seção relata a base teórica que fundamenta as teorias do ciclo de políticas públicas, de acordo com \citeonline{rua2013}.

A sua importância ressurgiu nas últimas décadas do século XX. 
Assim, estabelecendo regras e modelos que regem a decisão de políticas públicas em instituições, sendo fundamentada por fatores \cite{souza2006politicas}.  
O primeiro são as políticas restritivas de gastos baseadas pelo Consenso de Washington, o qual passaram a dominar a agenda de diversos países, como o Brasil.
O segundo foi a substituição das políticas keynesianas para políticas liberais \cite{resende2014escassez}.


\subsection{Surgimento da área de políticas públicas}


Os principais idealizadores da política pública são do século XX, os quais se destacam Harold Lasswell, Herbert A. Simon, Charles E. Lindblom e David Easton \cite{souza2006politicas}.

Em 1936 Harold Lasswell usou o termo policy analysis de forma a conciliar o conhecimento científico com a produção empírica dos governos e também de maneira a estabelecer o diálogo entre os cientistas sociais e os grupos de interesses. 
Então, em 1497 Herber A. Simon sustentou que a tomada de decisão é o centro vital da administração, por isto, deve se basear em processos decisórios considerando a lógica e a psicologia da escolha humana~\cite{rua2013}.

A área de políticas públicas se desenvolveu em campos de estudo a partir de 1950. 
A aplicação da Teoria dos sistemas por David Easton em 1953.
O livro The Policy Sciences de Lasswell e Daniel Lerner, que propôs o termo policy sciences, desdobra o processo de produção da política pública, sugerindo as etapas de informação, promoção, prescrição, invocação, aplicação, término e avaliação. 
Esses pensamento foram críticados por Charles E. Lindblom em 1959, que defendiam o processo político e decisório, constitui um processo interativo e complexo, sem início ou fim, no qual as relações de poder é o elemento crucial~\cite{rua2013}.

A década de 1960 foi marcada pelo Modelo de Análise Sistêmica por Easton em 1962, que resultou em polêmicas entre o pensamento elitista e a abordagem pluralista. Além disso, Easton definiu em 1965 a policies como produto da operação da politics~\cite{rua2013}.

As décadas seguintes tornaram a pauta para formação da agenda, recebendo contribuição de Richard Rose em 1973. 
Por fim, os estágios da policy cycle foram elaborados, as contribuições de Jeffrey Pressman, Aaron Wildavsky, Carol Weiss entre outros~\cite{rua2013}.

Conforme Lowi, as políticas públicas são regras formuladas por autoridades governamentais que expressam intenção de influenciar o comportamento dos cidadãos por intermédio da utilização de sanções positivas ou negativas. 
Portanto, os governos coagem os seus cidadãos por meio dessas sanções~\cite{rua2013}. 

Conclui-se que o pressuposto analítico visa a formulação científica e análise por pesquisadores independentes. 
O estudo das instituições considera fundamentais para limitar a tirania e as paixões inerentes à natureza humana, conforme a tradição de Madison. 
As organizações locais são as virtudes cívicas para promover o governo ideal, de acordo com a tradição de Paine e Tocqueville. 
Política pública é um ramo da ciência da política para entender como e por que os governos optam por determinadas ações \cite{souza2006politicas}.


\subsection{Definição de políticas públicas}


As decisões e análises sobre política pública implicam em responder às seguintes questões: quem, o quê, por quê e que diferença faz.
Assim, a política pública como o campo do conhecimento que busca, simultaneamente, colocar o governo em ação ou analisar essa ação (variável independente) e, quando necessário, propor mudanças no rumo ou curso dessa ação (variável dependente).
Se admitirmos que política pública é uma área que situa diversas unidades em totalidades organizadas, isso possui duas implicações \cite{souza2006politicas}. 
A primeira, ser uma área do conhecimento multidisciplinar contendo várias disciplinas, teorias e modelos analíticos.
A segunda, apesar dela ser uma área multidisciplinar, ela não carece de coerência teórica e metodológica.
Por fim, políticas públicas, após desenhadas e formuladas, desdobram-se em planos, programas, projetos, bases de dados ou sistema de informação e pesquisa. 
Então, quando postas em ação, são implementadas, então, submetidas a sistema de acompanhamento e avaliação.


\subsection{Papel dos governos}


O processo de definição de políticas públicas sociedades e Estados complexos como os constituídos ao mundo moderno estão mais próximos à perspectiva teórica daqueles que defendem que existe uma autonomia relativa do Estado, o que faz com que esse tenha um espaço próprio de atuação, embora permeável a influências externas e internas.
Tal autonomia relativa gera determinadas capacidades de criar condições para implementação de objetivos de políticas públicas, o qual depende de vários fatores e dos diferentes momentos históricos de cada localidade.
Apesar do reconhecimento de que outros segmentos que não os governos se envolvem na formulação de políticas públicas, tais como grupos de interesses e os movimentos nacionais e internacionais. Isto não interfere ou diminui a elaboração do governo na formulação de políticas públicas, ao menos não há comprovação empiricamente sobre a diminuição da participação do governo em virtude de segmentos não governamentais \cite{souza2006politicas}.


\subsection{Modelos de formulação e análise de políticas públicas}


Possui o objetivo de representar ou buscar a representação das políticas públicas, assim, explicar e entender as decisões do governo. 
Os modelos que serão listados são os seguintes: tipo de política pública, incrementalismo, ciclo da política pública, garbage can, coalizão de defesa, arenas sociais, equilíbrio interrompido e novo gerencialismo público.

O Tipo da Política Pública é desenvolvido por Theodor Lowi, defendendo que a política pública faz a política.
Então, cada tipo de política pública encontra diferentes formas de apoio e de rejeição e que disputas em torno de sua decisão passam por arenas diferenciadas, podendo assumir quatro formatos: política distributiva, beneficia toda a sociedade mediante aos recursos provenientes da coletividade como um todo; política regulatória, medidas imperativas que estabelecem condições por meio das quais podem e devem ser realizadas determinadas atividades ou admitir certos comportamentos; política redistributiva, compensar determinados grupos por intermédio de alocação de recursos de terceiros tendo o objetivo de realizar justiça entre os grupos, sendo assim um política que gera conflitos; e políticas constitutivas, determina as regras do governo e das suas políticas, ou seja, são as normas e os procedimentos sobre as quais deve ser formuladas e implementadas as demais políticas públicas~\cite{rua2013}.

O Incrementalismo idealizado por Lindblom, Caiden e Wildavsky por intermédio das pesquisas empíricas. 
Destarte, os recursos governamentais para entidades surgem de decisões marginais e incrementais que desconsideram mudanças políticas ou mudanças substantivas nos programas públicos. 
Por conseguinte, as decisões do governo seriam apenas incrementais e escassez das substantivas.
No entanto, enfraqueceu como o surgimento  do Consenso de Washington devido aos ajustes fiscais por esse sugerido, porém a visão de que decisões tomadas no passado constrangem decisões futuras e limitam a capacidade dos governos de adotar novas políticas públicas ou de reverter a rota das políticas atuais \cite{souza2006politicas}.

O ciclo da política pública foi o método que obteve diversas divisões que se diferenciam apenas gradualmente. 
Portanto, é comum a todas as propostas serem as fases da formulação, da implementação e do controle dos impactos das políticas públicas \cite{rua2013}.
Por conseguinte, tendo para alguns autores idealizaram uma subdivisão, como é o exemplo desse artigo que é constituído dos seguintes estágios: definição de agenda, identificação de alternativas, avaliação das opções, seleção das opções, implementação e avaliação.

Garbage Can surgiu em 1972 idealizado por Michael D. Cohen, James G. March e Johan P. Olsen, sendo um modelo alternativa à concepção do ciclo de políticas públicas, não possui precedência nem temporal, tampouco lógica entre os problemas para a formação da agenda  e as soluções~\cite{rua2013}.
Portanto, o modelo recolhe diversos tipos de problemas e soluções que são colocados pelos participantes à medida que eles aparecem.

A Coalizão de Defesa são subconjuntos de atores que são agregados de acordos com seus ideais, em geral, cada subsistema abriga ao menos duas coalizões de defesa.
Os membros de uma Coalizão de Defesa são atores formais e informais, situados em várias organizações em torno de seus objetivos comuns~\cite{rua2013}.
Embora o conceito desse modelo enfatiza o sistema de crenças e a horizontalização das relações entre os atores, sua estrutura argumentativa não apresenta elementos substanciais para contrapô-lo a nenhuma das teorias de distribuição do poder, conforme afirma~\citeonline{rua2013}.

As Arenas Sociais é determinado por empreendedores políticos, as quais determinam as circunstâncias para mapear o problema, pois é preciso que os indivíduos se convençam de que algo precisa ser feito, de modo que o governo note tais questões, tal fenômeno é dado por policy makers, podendo ser notados por meio da divulgação de indicadores, eventos e  feedback \cite{souza2006politicas}.
Portanto, o empreendedorismo político junto às suas ferramentas são cruciais para a sobrevivência e o sucesso da ideia e adicionar o problema na agenda pública.

O Equilíbrio Interrompido elaborado por Baumgartner e Jones em 1993, tendo como inspiração o modelo biológico e computacional, na medida em que o mundo é escasso, longo os recursos também são. 
Por conseguinte, o modelo representa as decisões de modo paralelo, assim, as mudanças são feitas de modo a suprir as necessidades dele, em períodos estável é adotado as reformas no modelo.  
Portanto, o modelo proposto é o ciclo de vida guiado pela estabilidade do governo \cite{souza2006politicas}.

O Novo Gerencialismo Público baseado pelo Consenso de Washington, que implementa políticas fiscais para restringir os gastos públicos e a meta de política pública seja efetiva.


\subsection{O papel das instituições na decisão e formulação de políticas públicas}


A Teoria da Escolha Racional questiona dois mitos, sendo o interesse dos indivíduos agregados gerarem ação coletiva; e outro baseado pela ação coletiva produzir necessariamente bens coletivos.
Além disso, depende das percepções subjetivas sobre as alternativas, suas consequências e avaliações dos seus possíveis resultados.
Embora, não desconsiderando a existência do cálculo racional e auto-interessado dos decisores, esses agem e se organizam de acordo com as regras e práticas socialmente  construídas, conhecidas antecipadamente e aceitas.

A Teoria da Escolha Pública segue o viés normativo para formular a política pública devido a situação como auto-interesse, informação incompleta, racionalidade limitada e captura das agências governamentais por interesses particulares \cite{souza2006politicas}.

O Neo-Institucionalismo é destacado pela razão do poder e dos recursos entres os grupos sociais é essencial para formulação de políticas públicas.
Destarte, baseia-se na política redistributiva, no qual beneficia determinado grupo em detrimento de outros.

Conclui que a busca unificar e integrar os conhecimentos sobre a política, a pública, a politics, a polity e as instituições que tomam tais medidas.
Por conseguinte, o objetivo analítico da política pública é relacionado a identificação do tipo de problema e como a política pública pode sanar tal problema.
Tendo o problema listado e adicionado a agenda as instituições modelam a a decisão e a implementação da política pública.   
Logo, a compreensão dos modelos de formulação e análise de políticas públicas são essenciais para a construção das agendas e do ciclo de política do governo, assim, resolver os conflitos e realizar negociações para implementar a política pública de maneira eficiente, eficaz e efetiva \cite{souza2006politicas}.

