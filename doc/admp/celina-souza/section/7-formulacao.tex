\section{O papel das instituições/regras na decisão e formulação de políticas públicas}


A teoria da escolha racional questiona dois mitos, sendo o interesse dos indivíduos agregados gerarem ação coletiva; e outro baseado pela ação coletiva produzir necessariamente bens coletivos.
Além disso, depende das percepções subjetivas sobre as alternativas, suas consequências e avaliações dos seus possíveis resultados.
Embora, não desconsiderando a existência do cálculo racional e auto-interessado dos decisores, esses agem e se organizam de acordo com as regras e práticas socialmente  construídas, conhecidas antecipadamente e aceitas.

A teoria da escolha pública segue o viés normativo para formular a política pública devido a situação como auto-interesse, informação incompleta, racionalidade limitada e captura das agências governamentais por interesses particulares.

O neo-institucionalismo é destacado pela razão do poder e dos recursos entres os grupos sociais é essencial para formulação de políticas públicas.
Destarte, baseia-se na política redistributiva, no qual beneficia determinado grupo em detrimento de outros.
