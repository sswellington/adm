\section{Modelos de formulação e análise de políticas públicas}


Possui o objetivo de representar ou buscar a representação das políticas públicas, assim, explicar e entender as decisões do governo. 
Os modelos que serão listados são os seguintes: tipo de política pública, incrementalismo, ciclo da política pública, garbage can, coalizão de defesa, arenas sociais, equilíbrio interrompido e novo gerencialismo público.

O Tipo da Política Pública é desenvolvido por Theodor Lowi, defendendo que a política pública faz a política.
Então, cada tipo de política pública encontra diferentes formas de apoio e de rejeição e que disputas em torno de sua decisão passam por arenas diferenciadas, podendo assumir quatro formatos: política distributiva, beneficia toda a sociedade mediante aos recursos provenientes da coletividade como um todo; política regulatória, medidas imperativas que estabelecem condições por meio das quais podem e devem ser realizadas determinadas atividades ou admitir certos comportamentos; política redistributiva, compensar determinados grupos por intermédio de alocação de recursos de terceiros tendo o objetivo de realizar justiça entre os grupos, sendo assim um política que gera conflitos; e políticas constitutivas, determina as regras do governo e das suas políticas, ou seja, são as normas e os procedimentos sobre as quais deve ser formuladas e implementadas as demais políticas públicas~\cite{rua2013}.

O Incrementalismo idealizado por Lindblom, Caiden e Wildavsky por intermédio das pesquisas empíricas. 
Destarte, os recursos governamentais para entidades surgem de decisões marginais e incrementais que desconsideram mudanças políticas ou mudanças substantivas nos programas públicos. 
Por conseguinte, as decisões do governo seriam apenas incrementais e escassez das substantivas.
No entanto, enfraqueceu como o surgimento  do Consenso de Washington devido aos ajustes fiscais por esse sugerido, porém a visão de que decisões tomadas no passado constrangem decisões futuras e limitam a capacidade dos governos de adotar novas políticas públicas ou de reverter a rota das políticas atuais.

O ciclo da política pública foi o método que obteve diversas divisões que se diferenciam apenas gradualmente. 
Portanto, é comum a todas as propostas serem as fases da formulação, da implementação e do controle dos impactos das políticas públicas \cite{rua2013}.
Por conseguinte, tendo para alguns autores idealizaram uma subdivisão, como é o exemplo desse artigo que é constituído dos seguintes estágios: definição de agenda, identificação de alternativas, avaliação das opções, seleção das opções, implementação e avaliação.

Garbage Can surgiu em 1972 idealizado por Michael D. Cohen, James G. March e Johan P. Olsen, sendo um modelo alternativa à concepção do ciclo de políticas públicas, não possui precedência nem temporal, tampouco lógica entre os problemas para a formação da agenda  e as soluções~\cite{rua2013}.
Portanto, o modelo recolhe diversos tipos de problemas e soluções que são colocados pelos participantes à medida que eles aparecem.


Coalizão de defesa são subconjuntos de atores que são agregados de acordos com seus ideais, em geral, cada subsistema abriga ao menos duas coalizões de defesa.
Os membros de uma Coalizão de Defesa são atores formais e informais, situados em várias organizações em torno de seus objetivos comuns~\cite{rua2013}.
Embora o conceito desse modelo enfatiza o sistema de crenças e a horizontalização das relações entre os atores, sua estrutura argumentativa não apresenta elementos substanciais para contrapô-lo a nenhuma das teorias de distribuição do poder, conforme afirma~\citeonline{rua2013}.


Arenas sociais é determinado por empreendedores políticos, as quais determinam as circunstâncias a se transforme em um problema, pois é preciso que os indivíduos se convençam de que algo precisa ser feito, de modo que o governo note tais questões, tal fenômeno é dado por policy makers, podendo ser notados pela seguinte ferramentas por divulgação de indicadores, eventos e  feedback.
Portanto, o empreendedorismo político junto às suas ferramentas são cruciais para a sobrevivência e o sucesso da ideia e adicionar o problema na agenda pública.


Equilíbrio interrompido elaborado por Baumgartner e Jones em 1993, tendo como inspiração o modelo biológico e computacional, na medida em que o mundo é escasso, longo os recursos também são. 
Por conseguinte, o modelo representa as decisões de modo paralelo, assim, as mudanças são feitas de modo a suprir as necessidades dele, em períodos estável é adotado as reformas no modelo.  
Logo, o modelo proposto é o ciclo de vida determinado pela estabilidade.


Novo gerencialismo público baseado no Consenso de Washington, que implementa políticas fiscais para restringir os gastos públicos e que a meta de política pública seja efetiva.
