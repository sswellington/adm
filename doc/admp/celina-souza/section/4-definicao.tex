\section{Definição de políticas públicas}


As decisões e análises sobre política pública implicam em responder às seguintes questões: quem, o quê, por quê e que diferença faz.
Assim, a política pública como o campo do conhecimento que busca, simultaneamente, colocar o governo em ação ou analisar essa ação (variável independente) e, quando necessário, propor mudanças no rumo ou curso dessa ação (variável dependente).
Se admitirmos que política pública é uma área que situa diversas unidades em totalidades organizadas, isso possui duas implicações. 
A primeira, ser uma área do conhecimento multidisciplinar contendo várias disciplinas, teorias e modelos analíticos.
A segunda, apesar dela ser uma área multidisciplinar, ela não carece de coerência teórica e metodológica.
Por fim, políticas públicas, após desenhadas e formuladas, desdobram-se em planos, programas, projetos, bases de dados ou sistema de informação e pesquisa. 
Então, quando postas em ação, são implementadas, então, submetidas a sistema de acompanhamento e avaliação.
