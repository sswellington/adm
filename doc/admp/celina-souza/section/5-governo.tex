\section{Papel dos governos}


O processo de definição de políticas públicas sociedades e Estados complexos como os constituídos ao mundo moderno estão mais próximos à perspectiva teórica daqueles que defendem que existe uma autonomia relativa do Estado, o que faz com que esse tenha um espaço próprio de atuação, embora permeável a influências externas e internas.
Tal autonomia relativa gera determinadas capacidades de criar condições para implementação de objetivos de políticas públicas, o qual depende de vários fatores e dos diferentes momentos históricos de cada localidade.
Apesar do reconhecimento de que outros segmentos que não os governos se envolvem na formulação de políticas públicas, tais como grupos de interesses e os movimentos nacionais e internacionais. Isto não interfere ou diminui a elaboração do governo na formulação de políticas públicas, ao menos não há comprovação empiricamente sobre a diminuição da participação do governo em virtude de segmentos não governamentais.
