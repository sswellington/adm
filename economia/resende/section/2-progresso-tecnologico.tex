\section{O modelo de referência movido a progresso tecnológico}

O modelo de Solow-Swan, ou modelo referência do crescimento defende que o quanto mais 
se poupa e se investe, mais se cresce, porém 
atingida a relação capital/produto de equilíbrio de longo prazo, o crescimento se torna independente da taxa de poupança e de investimento, de modo a depender apenas do progresso tecnológico.

Fatores explicativos do progresso tecnológico estão relacionados à educação e à pesquisa. 
Apesar da Revolução da Informática o crescimento econômico e populacional deverão se estabilizar dentro de algumas décadas.
Entretanto, isto não implica em o declínio da qualidade de vida, ao contrário disso, poderá continuar a melhorar, somente produção de bens que não possui viabilidade continuar a crescer indefinidamente.
