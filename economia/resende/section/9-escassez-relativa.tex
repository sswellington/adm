\section{O desafio da Escassez Relativa}

A economia capitalista do século XX foi propensa ao desequilíbrio, que foi ajustada pelo remédio keynesiano, no qual foi suficiente para garantir o crescimento até o final daquele século.
No século XXI, a pauta é garantir que todos tenham acesso ao produzido, se tornou a questão central do mundo contemporâneo e com a abundância e menos perspectiva de melhora, a desigualdade fica menos tolerável.
Não obstante, a perspectiva de congelar, ou mesmo de agravar, a desigualdade da renda e da riqueza entre os países.
A Revolução da informática mantém a produtividade do capital, porém não a demanda do emprego.
Assim, contribuindo para a concentração de renda e da riqueza, por conseguinte passamos da era da escassez absoluta para a da escassez relativa.
