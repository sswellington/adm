\section{Introdução}

Resumo do Capítulo 1 - Introdução do Livro Um Toque de Clássicos: Marx, Durkheim e Weber de \citeonline{quintaneiro2003toque}.


A sociologia emerge em meados do século XIX, tendo objetivo de entender as mudanças sociais, políticas e econômicas da sociedade.
O século XVI serviu de amostragem para o estudo, em virtude às correntes de pensamentos oriundas do racionalismo, empirismo e iluminismo, estabelecendo o período Renascentista.
Tal período foi marcado por instabilidade e crises em diversos setores da sociedade, como cultural, moral e material.
Portanto, a sociologia surgiu para interpretar os eventos ocorridos nessa cronologia.



% \subsection{Mudanças resultantes da industrialização}

As mudanças resultantes da industrialização aconteceram de maneira súbita e quase imperceptível para os que presenciaram o evento histórico.
Tal intervalo de tempo foi marcado pelo fim do Renascimento, que sucedeu o período Medievo.
A ruptura ocorreu com o êxodo rural em busca de melhores condições de vida e suprir as demandas industriais, porém ocasionam diversos problemas habitacionais e sociais, visto que as cidades não possuíam infraestrutura para comportar a demanda populacional.

O ponto de inflexão foi o século XVIII que adotou políticas sanitárias, incremento da produção do setor alimentício e industrial, de forma que reduziu a taxa de mortalidade. 
Além disso, aumentou a expectativa de vida e populacional, porém os cidadãos enfrentam condições precárias de trabalho.

O advento da modernidade estabeleceu a padronização de compromisso baseado pela percepção da hora, reconhecimento da infância e o direito ao casamento por escolha mútua.
Entretanto, existia uma lacuna entre o núcleo familiar da nobreza aos populares.
Portanto, a sociologia se cristalizou ao entender os motivos prováveis que levaram a ruptura nas relações sociais ao decorrer desse período.



% \subsection{Antecedentes intelectuais da sociologia}

A Antiguidade Clássica era fundada em corrente de pensamento individualista.
O primeiro embate a essa corrente de pensamento foi a Reforma Protestante, no qual defendeu que o destino dos homens a eles pertencem, removendo assim a convicção que a Igreja era soberana.
Resultando diretamente na educação das universidades católicas que passaram a adotar as ciências naturais e exatas.

A Revolução Francesa desafiou o poder autoritário da monarquia, de forma a contribuir para liberdade, fraternidade e igualdade.
Sendo os pilares para a democracia. 
A série de ideias centradas na razão ganhou força, tornando-se protagonista, dando assim origem ao Iluminismo. 

Montesquieu contribuiu para a fundação do Estado moderno por intermédio da concepção da Teoria da Separação de Poderes e as Concepção de Pesos e Contrapesos.
Jean-Jacques Rousseau parte da concepção de ausência de desigualdade.
Defende que o homem é livre, só que na formação da sociedade e das leis, ele perde a sua liberdade, assim, se torna escravo do material, de maneira a ser o primeiro progresso de desigualdade.
